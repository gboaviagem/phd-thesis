\usepackage{mathtools}
\usepackage{booktabs}
\usepackage{amsmath,amsthm,amsfonts,cite,color,enumerate,subfig}
\usepackage{mathrsfs}
\usepackage{enumitem}
\usepackage{nicefrac}
\usepackage{subfig}
\usepackage{algpseudocode}
\usepackage{lscape}
\usepackage{framed}
\usepackage{cancel} % Use \usepackage[thicklines]{cancel} for thicker strokes

\definecolor{cancelgray}{gray}{0.5}
\renewcommand{\CancelColor}{\color{cancelgray}}

\newtheorem{theorem}{Theorem}
\newtheorem{example}{Example}
\newtheorem{corollary}{Corollary}
\newtheorem{definition}{Definition}
\newtheorem{algorithm}{Algorithm}
\newtheorem{lem}[theorem]{Lemma}

% NEW COMMANDS
\renewcommand{\mod}[1]{\  (\mathrm{mod} \ #1)}
\newcommand{\qi}{\boldsymbol{i}}
\newcommand{\qphi}{\boldsymbol{\phi}}
\newcommand{\qj}{\boldsymbol{j}}
\newcommand{\qk}{\boldsymbol{k}}
\newcommand{\qmu}{\boldsymbol{\mu}}
\newcommand{\qnu}{\boldsymbol{\nu}}
\newcommand{\qV}{\boldsymbol{V}}
\newcommand{\sen}{\mathrm{\, sen \,}}
\newcommand{\red}[1]{{\color{red}#1}}
\DeclareRobustCommand{\rchi}{{\mathpalette\irchi\relax}}
\newcommand{\irchi}[2]{\raisebox{\depth}{\Large $#1\chi$}} % inner command, used by \rchi

% ISOMORPHISM SYMBOL
\makeatletter
\newcommand*{\isomorphism}{%
	\mathrel{%
		\mathpalette\@isomorphism{}%
	}%
}
\newcommand*{\@isomorphism}[2]{%
	% Calculate the amount of moving \sim up as in \simeq
	\sbox0{$#1\simeq$}%
	\sbox2{$#1\sim$}%
	\dimen@=\ht0 %
	\advance\dimen@ by -\ht2 %
	%
	% Compose the two symbols
	\sbox0{%
		\lower1.9\dimen@\hbox{%
			$\m@th#1\relbar\isomorphism@joinrel\relbar$%
		}%
	}%
	\rlap{%
		\hbox to \wd0{%
			\hfill\raise\dimen@\hbox{$\m@th#1\sim$}\hfill
		}%
	}%
	\copy0 %
}
\newcommand*{\isomorphism@joinrel}{%
	\mathrel{%
		\mkern-3.4mu %
		\mkern-1mu %
		\nonscript\mkern1mu %
	}%
}
\makeatother
