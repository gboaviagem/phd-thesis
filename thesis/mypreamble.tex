\usepackage{mathtools}
\usepackage{booktabs}
\usepackage{amsmath,amsthm,amsfonts,cite,color,enumerate,subfig}
\usepackage{enumitem}
\usepackage{nicefrac}
\usepackage{subfig}

\newtheorem{theorem}{Teorema}
\newtheorem{example}{Exemplo}
\newtheorem{corollary}{Corol\'ario}
\newtheorem{definition}{Defini\c c\~ao}
\newtheorem{lem}[theorem]{Lemma}
\newtheorem{cor}[theorem]{Corollary}

% NEW COMMANDS
\renewcommand{\mod}[1]{\  (\mathrm{mod} \ #1)}
\newcommand{\qi}{\boldsymbol{i}}
\newcommand{\qj}{\boldsymbol{j}}
\newcommand{\qk}{\boldsymbol{k}}
\newcommand{\qmu}{\boldsymbol{\mu}}
\newcommand{\qnu}{\boldsymbol{\nu}}
\newcommand{\qV}{\boldsymbol{V}}
\newcommand{\sen}{\mathrm{\, sen \,}}
\newcommand{\red}[1]{{\color{red}#1}}

% ISOMORPHISM SYMBOL
\makeatletter
\newcommand*{\isomorphism}{%
	\mathrel{%
		\mathpalette\@isomorphism{}%
	}%
}
\newcommand*{\@isomorphism}[2]{%
	% Calculate the amount of moving \sim up as in \simeq
	\sbox0{$#1\simeq$}%
	\sbox2{$#1\sim$}%
	\dimen@=\ht0 %
	\advance\dimen@ by -\ht2 %
	%
	% Compose the two symbols
	\sbox0{%
		\lower1.9\dimen@\hbox{%
			$\m@th#1\relbar\isomorphism@joinrel\relbar$%
		}%
	}%
	\rlap{%
		\hbox to \wd0{%
			\hfill\raise\dimen@\hbox{$\m@th#1\sim$}\hfill
		}%
	}%
	\copy0 %
}
\newcommand*{\isomorphism@joinrel}{%
	\mathrel{%
		\mkern-3.4mu %
		\mkern-1mu %
		\nonscript\mkern1mu %
	}%
}
\makeatother
