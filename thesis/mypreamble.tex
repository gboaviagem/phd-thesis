%\usepackage{color}
%%\usepackage[dvipsnames]{xcolor}
%\usepackage{booktabs}
%\usepackage{longtable}
%\usepackage{bm}
%\usepackage{upgreek}
%\usepackage{amsmath}
%\usepackage{amsthm}
%\usepackage{amsfonts}
%\usepackage{amssymb}
%\usepackage{multirow}
%
%\usepackage{bibentry}
%%\usepackage{setspace}
%%\singlespacing
%%\onehalfspacing
%%\doublespacing
%%\setstretch{1.2}
%\usepackage{emptypage} %that will empty all blank pages created by \cleardoublepage
%\usepackage{url}	
%\usepackage[driverfallback=dvipdfm]{hyperref}
%\usepackage[font={small}, labelfont={bf}, margin=1cm]{caption}
%\usepackage{subfig}
\usepackage{enumitem}
%\usepackage{graphicx}
%\usepackage{xcolor}
%\usepackage{url} % Permite exibição de sites de forma organizada
\usepackage{nicefrac}
%\usepackage{cite}
%%\usepackage{parskip}
%
%
%% FONT
%%\usepackage{times}
%%\usepackage{tgtermes}
%%\usepackage{palatino}
%
%%\usepackage{epigraph}
%%\setlength{\epigraphwidth}{.6\textwidth}
%
\usepackage{mathtools}
\usepackage{booktabs}
\usepackage{amsmath,amsthm,amsfonts,cite,color,enumerate,subfig}
%\newtheorem{theorem}{Teorema}
%\newtheorem{example}{Exemplo}
%\newtheorem{corollary}{Corol\'ario}
%\newtheorem{definition}{Defini\c c\~ao}
%\newtheorem{lem}[thm]{Lemma}
%\newtheorem{cor}[thm]{Corollary}

%\newtheorem{teorema}{Teorema}
%\newtheorem{lema}{Lema}
%\newtheorem{corolario}[theorem]{Corol\'ario}
\usepackage{subfig}
%
%
% NEW COMMANDS
\renewcommand{\mod}[1]{\  (\mathrm{mod} \ #1)}
%\newcommand{\ord}[1]{\mathrm{ord}(#1)}
\newcommand{\qi}{\boldsymbol{i}}
\newcommand{\qj}{\boldsymbol{j}}
\newcommand{\qk}{\boldsymbol{k}}
\newcommand{\qmu}{\boldsymbol{\mu}}
\newcommand{\qnu}{\boldsymbol{\nu}}
\newcommand{\qV}{\boldsymbol{V}}
\newcommand{\sen}{\mathrm{\, sen \,}}
\newcommand{\red}[1]{{\color{red}#1}}

% ISOMORPHISM SYMBOL

\makeatletter
\newcommand*{\isomorphism}{%
	\mathrel{%
		\mathpalette\@isomorphism{}%
	}%
}
\newcommand*{\@isomorphism}[2]{%
	% Calculate the amount of moving \sim up as in \simeq
	\sbox0{$#1\simeq$}%
	\sbox2{$#1\sim$}%
	\dimen@=\ht0 %
	\advance\dimen@ by -\ht2 %
	%
	% Compose the two symbols
	\sbox0{%
		\lower1.9\dimen@\hbox{%
			$\m@th#1\relbar\isomorphism@joinrel\relbar$%
		}%
	}%
	\rlap{%
		\hbox to \wd0{%
			\hfill\raise\dimen@\hbox{$\m@th#1\sim$}\hfill
		}%
	}%
	\copy0 %
}
\newcommand*{\isomorphism@joinrel}{%
	\mathrel{%
		\mkern-3.4mu %
		\mkern-1mu %
		\nonscript\mkern1mu %
	}%
}
\makeatother
% =========================================