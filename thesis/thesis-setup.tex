
% **************************************************
% Files' Character Encoding
% **************************************************
\PassOptionsToPackage{utf8}{inputenc}
\usepackage{inputenc}



% **************************************************
% Information and Commands for Reuse
% **************************************************
\newcommand{\thesisTitle}{title}
\newcommand{\thesisName}{author}
\newcommand{\mypriorstudies}{authorgraduation}
\newcommand{\mypriorstudiesPlace}{Universidad Polit\'ecnica de Madrid}
\newcommand{\thesisSubject}{Doctoral Dissertation}
\newcommand{\thesisDate}{December xx, 2018}
\newcommand{\thesisVersion}{My First Draft}

\newcommand{\thesisFirstReviewer}{John Smith}
\newcommand{\thesisFirstReviewerUniversity}{\protect{Clean Thesis Style University}}
\newcommand{\thesisFirstReviewerDepartment}{Department of Clean Thesis Style}

\newcommand{\thesisSecondReviewer}{John Doe}
\newcommand{\thesisSecondReviewerUniversity}{\protect{Clean Thesis Style University}}
\newcommand{\thesisSecondReviewerDepartment}{Department of Clean Thesis Style}

\newcommand{\thesisFirstSupervisor}{Mark Scott}
\newcommand{\thesisSecondSupervisor}{Tom Parker}

\newcommand{\thesisUniversity}{\protect{Universidad Polit\'ecnica de Madrid}}
\newcommand{\thesisUniversityDepartment}{Department}
\newcommand{\thesisUniversityInstitute}{Escuela T\'ecnica Superior de Ingenieros Industriales}
\newcommand{\thesisUniversityGroup}{Escuela T\'ecnica Superior de Ingenieros Industriales}
\newcommand{\thesisUniversityCity}{Madrid}
\newcommand{\thesisUniversityStreetAddress}{C\textbackslash  Jos\'e Guti\'errez Abascal 2}
\newcommand{\thesisUniversityPostalCode}{28006}

% **************************************************
% Debug LaTeX Information
% **************************************************
%\listfiles


% **************************************************
% Load and Configure Packages
% **************************************************




%Emilio
\newcommand{\bs}[1]{\boldsymbol{#1}}
\newcommand{\mb}[1]{\mathbf{#1}}
\usepackage{amsmath}
\usepackage{physics}
\usepackage{cancel}
%\usepackage{subfigure}
\usepackage{gensymb}
\usepackage{booktabs}
\usepackage{multirow}
\newcommand{\nn}{\nonumber}
\usepackage{bm}
\usepackage[super]{nth}
\usepackage{lscape}
\usepackage{dsfont}
\usepackage{isotope}
\usepackage{geometry}
\usepackage{enumitem}
\usepackage{subcaption}
\usepackage{longtable}
\usepackage{multirow}
\usepackage{afterpage}
\usepackage{changepage}
\usepackage[title]{appendix}
\usepackage{breakcites}
%End Emilio

% **************************************************
% Load and Configure Packages
% **************************************************
\usepackage[utf8]{inputenc}		% defines file's character encoding
\usepackage[english]{babel} % babel system, adjust the language of the content
\PassOptionsToPackage{% setup clean thesis style
	figuresep=colon,%
	hangfigurecaption=false,%
	hangsection=true,%
	hangsubsection=true,%
	colorize=bw,% full, reduced, bw
	colortheme=bluemagenta,%
	bibsys=biber,%
	bibfile=mybib,%
	bibstyle=authoryear,%
    bibliographystyle=ieeetr
}{cleanthesis}
\usepackage{cleanthesis}

%Emilio
%This allows creating diagrams and drawing arrows in the tables
\usepackage{tikz}
\usetikzlibrary{tikzmark,matrix,arrows.meta,trees,shapes,positioning,calc,shadows,fit}
\tikzset{
	centered/.style = { align=center, anchor=center },
	empty/.style = { font=\rmfamily\Large, centered, text width=2cm },
	boxSolidLine/.style = { fill=white, draw=black, thick, centered },
	boxSolidLine7ex/.style = { fill=white, draw=black, thick, centered ,minimum width = 8ex, minimum height = 8ex},
	invisible/.style = { fill=white, draw=white, text=white },
	hexagonSolidLine/.style = { fill=white, draw=black, thick, regular polygon, regular polygon sides = 6, centered },
	octogonSolidLine/.style = { fill=white, draw=black, thick, regular polygon, regular polygon sides = 8, centered },
	ellipseSolidLine/.style = { fill=white, draw=black, thick, ellipse, centered },
	decision/.style = {diamond, thick, draw=black,align=center,inner sep=1pt, aspect=2},
	boxDashedLine/.style = { fill=white, draw=black, thick, dashed, centered },
	boxDottedLine/.style = { fill=white, draw=black, thick, dotted, centered },
	shadowbox/.style={boxSolidLine, double copy shadow={shadow xshift=-1.5ex, shadow yshift=1.5ex}},
	result/.style = { fill=black!20, centered},
	blank/.style = { fill=white, left,centered,opacity=0,text opacity=1},
	arrow/.style = { thick, color=black, ->, >=Triangle},
	arrowDouble/.style = { thick, color=black, <->, >=Triangle},
	line/.style = { thick, color=black},
	arrowshadow/.style = { thick, color=black, ->, >=Triangle,double copy shadow={thick, shadow xshift=-1.5ex, shadow yshift=1.5ex}},
}

%Lo siguiente es para ajustar el tamaño de letra de las secciones con formato verbatim
\usepackage{fancyvrb}
%End Emilio

\hypersetup{					% setup the hyperref-package options
	pdftitle={\thesisTitle},	% 	- title (PDF meta)
	pdfsubject={\thesisSubject},% 	- subject (PDF meta)
	pdfauthor={\thesisName},	% 	- author (PDF meta)
	plainpages=false,			% 	-
	colorlinks=false,			% 	- colorize links?
	pdfborder={0 0 1},			% 	- Para quitar el recuadro en los links poner 0 0 0 
	breaklinks=true,			% 	- allow line break inside links
	bookmarksnumbered=true,		%
	bookmarksopen=true			%
}

\usepackage[acronym,nonumberlist,nomain]{glossaries}

