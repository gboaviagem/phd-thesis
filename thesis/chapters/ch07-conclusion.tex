\chapter{Conclusion}
\label{ch:conclusion}

\begin{quotation}
    \itshape
    Quaternions came from Hamilton after his really good work had been done; and, though beautifully ingenious, have been an unmixed evil to those who have touched them in any way [...].

    \noindent --- Lord Kelvin, letter to Hayward, 1892, as cited in \parencite{altmann1989hamilton}.
\end{quotation}

This thesis has tackled the problem of extending the basic tools of graph signal processing by changing the subjacent algebra in which signals and edge weights were defined, moving from the complex to the quaternion numbers. The exploration has also led to contributions on fractional linear operators, specifically proposing the fractional graph shift and a new way of computing the fractional QDFT, the latter having been used on an encryption scheme of color images in the form of a multiparametric transform.

The opening quote of this chapter is intended to be read in a light and ironic tone. The only ``evil'' the quaternions have bestowed on this thesis -- if we can humorously declare this way --  was the struggles with understanding the intricacies of its algebra and application on signal processing. But instead of causing any harm, the journey has borne fruit. And has done so in three fronts, let us recall the main contributions in each of them.

In the proposed fractional shift operator, it was demonstrated how it approximates  the classical ideal interpolating filter in the case of ring graphs and how it can be implemented as an LSI graph filter, even for arbitrary graphs. Also, it was shown how the operator may be used in filter design to improve the frequency response with the same filter length.

In the study of the fractionalization of the QDFT, it was demonstrated how this transform share symmetric eigenvectors with the DFT, what allowed for the construction of an orthogonal eigenbasis of the QDFT and subsequent definition of its fractional (and even multiparametric) version. Since quaternions possess 4 real-valued components, an application of the 2D-MFrQDFT was proposed in the form of a holistic encryption scheme for color images with opacity layer, which was shown to provide satisfactorily large key space and key sensitivity.

Finally, the proposition of quaternion graph signal processing brought some interesting results. For instance, it was stated that a way to enforce a diagonalizable graph adjacency matrix is by making it Hermitian, which will provide also real-valued eigenvalues and a unitary QGFT matrix. During the examples provided, it was shown a method for graph inference which extracts the phase of the difference in adjacent samples and their geographic distance to build the edge weights, as well as a way of designing FIR LSI filters using QLMS optimization. All computations in the aforementioned examples were performed using a new open-source Python package, released as a Github repository for peer-review and free reuse.

\section{Challenges and future works}

Applying the quaternion algebra into GSP is an endeavor which is far from complete. The \textit{foundations} of QGSP have been laid, but there are certainly challenges ahead and future work opportunities. Starting from the computational cost of QGSP algorithms, there is room for improvement in the efficiency of certain steps of signal frequency analysis. As mentioned in Section \ref{subsec:inversion}, the closed formula for quaternion matrix inversion presented in \parencite{cohen1999quaternionic} has potential to be used to accelerate the QGFT matrix calculation, specifically when computing the inverse of the eigenvector matrix when the QGSO is not Hermitian.

Another improvement in QGSP which may be explored is using better versions of the QLMS algorithm \parencite{ogunfunmi2015adaptive} in the design of LSI filters, aiming at faster convergence or computational efficiency. Similarly, QGSP may benefit from existing adaptive graph filtering techniques, such as using the Normalized LMS for graph signals \parencite{spelta2020normalized} to design quaternion graph filters.

The contributions on fractional linear operators also leave space for future works. Regarding the fractional graph shift, an interesting thread of investigation is to find possible relationships between the FrGSO and mathematical tools for fractional diffusion in networks, since fractional diffusion has already been used to model certain phenomena that allow long-range interactions and are non-local in nature~\parencite{ilic2005,riascos2014,estrada2021,antil2021}. In the fractional QDFT front, specifically regarding the use of the MFrQDFT for color image encryption, the interested cryptography researcher could investigate how to embed this system in real-world scenarios, or focus on other hypercomplex-based transform to expand our work, ofr instance using some hypercomplex image moments, such as ternary radial harmonic Fourier moments and quaternion polar harmonic moments \parencite{wang2019ternary,wang2018quaternion}, to perform color image encryption.
