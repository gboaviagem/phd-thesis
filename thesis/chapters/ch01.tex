\chapter{Introduction}
\label{ch:Intro}

Signal processing is a field which often interweaves pure mathematics and engineering. One of its concerns is the \textit{representation} of signals (functions) and how different representations may be explored to better manipulate such signals. The spectral analysis via Fourier and similar transforms, for instance, aims to project the signal onto a domain in which the energy support is more compact (compression), or in which some frequencies are easier to be removed (filtering), or yet in which some relevant features may be created (feature engineering for machine learning problems), among others \cite{oppenheim1999discrete, rabiner2010theory, graf2015features, vergin1999generalized}.

One way to explore different signal representations is to alter the algebra over which its sample values are defined. That is the case with number-theoretic transforms \cite{lima2011finite,blahut2010fast,pedrouzo2017number,chandra2014exact}, which deal only with finite algebraic structures instead of the usual complex (e.g. Fourier transform) or real fields (e.g. cosine transform). As the underlying algebra is extended, e.g. going from $ \mathbb{R} $ to $ \mathbb{C} $, it is possible to process signals with more information per sample. Such was the motivation behind Sangwine's \cite{sangwine1996fourier} discrete version of a family of bidimensional transforms over the \textit{quaternions}, previously created by Ell \cite{ell1993quaternion}: to exploit this class of hypercomplex numbers with four real components to perform holistic color image processing. Upon mapping each color channel inside a pixel into an imaginary component of a quaternion number, the color image may be processed as a single 2D signal -- instead of three, one per color channel. The holistic (as opposed to separate) processing of the color signal allows to explore the correlation and coupling between the channels, and this is generally the advantage intended when using quaternion signal processing to handle three- or four-dimensional data \cite{took2008quaternion}. Computationally, the spectral decomposition requires only two complex discrete Fourier transforms (DFTs), instead of three (one per color channel). Ever since, quaternion transforms have been employed not only on color image processing \cite{ell2007hypercomplex,chen2018quaternion,li2013quaternion,evans2000hypercomplex}, but also on other tasks such as bivariate signal analysis \cite{flamant2017spectral,flamant2017time,flamant2018complete}.

Algebra extensions change the signal representation by looking at the signal \textit{samples} but, reminding the signal is nothing more than a function, one may also shift the attention to the \textit{domain}. In fact, such a reflection has led to the creation of graph signal processing (GSP). This field, which emerged in the last decade, is concerned with extending the concepts of classical discrete signal processing (DSP) to the case in which the signal samples \textit{reside in graph vertices}. It converges to the classical case of discrete-time domain when using a path or ring graph -- depending on the graph shift operator (GSO) at hand, as will soon be clear -- with equally weighted edges. The \textit{signal}, therefore, is a mapping from the vertex set to the real (or complex) numbers, and the fixed sampling frequency is represented as graph edges having the same weights (i.~e., as if the samples were at fixed ``distances'', one could say). GSP deals with the consequences of conecting more vertices in this graph, moving from the queue-shaped domaing into an arbitrary network. Many definitions from classical digital signal processing (DSP) have already found their way into GSP, thanks to the contributions of many scholars who proposed ways to implement filtering \cite{sandryhaila2013filters}, linear transforms \cite{sandryhaila2013gft,sardellitti2017graph}, interpolation \cite{segarra2015interpolation}, sampling theorem on graphs \cite{wang2015generalized,chen2016signal,tsitsvero2016signals}, among others.

The endeavour to experiment with different algebras or domain topologies has primarily theoretical motivations, but often reveal new solutions to real world problems. Taking GSP for instance, its applications spread from coding \cite{su2017graph} and light field image super resolution schemes \cite{rossi2017graph}, to regulatory genetic networks, in which graph-based methods have already improved three state-of-the-art network inference schemes \cite{pirayre2015brane,pirayre2017brane}, also reaching recommender systems, both collaborative filtering and content-based, using regularization of graph total variation \cite{benzi2016song}, semi-supervised learning through adaptive graph filters \cite{chen2014semi}, community detection using graph wavelets \cite{tremblay2014graph}, among others. The latter examples hint at the intersection between GSP and the fields of data science and machine learning -- well established and highly active, both in Academia and industry --, to the point where we can join Benjamin Ricaud and say that \emph{Fourier could very well be, today, a data scientist} \cite{ricaud2019fourier}.

As happened to GSP, the study of quaternion-valued signal processing has also found applications. Beyond the already mentioned use in manipulating color image and bivariate signals, the work on adaptive quaternion filters has been useful, for example, in wind profile prediction \cite{jiang2014general}. Or yet the remarkable union of graphs and quaternions in the work by \cite{zhang2019quaternion} -- the only one the author is aware of --, in which quaternion embeddings are used to form a low dimensional representation of entities and relations in knowledge graphs.

In this context, the current thesis aims to tackle the following problem: is it possible to perform signal processing having quaternion-valued samples residing on graphs with quaternion-weighted edges? What limitations and new possibilites may arise when exploring what we may call \textit{quaternion graph signal processing} (QGSP)? That is the spark which ignited the studies and results presented in the following chapters.

As it will become clear to the reader, however, the pathway to the basis of QGSP was not a straight line. Sometimes what seemed to be dead ends emerged, and during the reflections upon ways to bypass them, other interesting results started to appear. For example, it was certain from the beginning that the problem of eigendecomposing quaternion matrices would be at the heart of QGSP, since the very definition of a graph signal spectrum requires the eigenvalues and eigenvectors of a GSO (e.g. the adjacency matrix), which is a quaternion matrix. Since it took a while to make progress with general graph matrices, it seemed promising to investigate the quaternion discrete Fourier transform (QDFT); after all, ring graphs should have the graph Fourier transform (GFT) and the classical discrete Fourier transform (DFT) coinciding. By looking at the QDFT matrix, however, a new way to diagonalize it was found and a new multiparametric fractional QDFT was proposed. When shifting the focus to the graph domain, revisiting the study of graph shift operators and their spectral decomposition, interesting results regarding the fractional GSO were discovered. For instance, this new operator acts as a graph filter, and its polynomial representation was fully determined, demonstrating that it can be implemented as a linear and shift-invariant (LSI) graph filter. These unforeseen results are featured in the thesis, as they paved the way for the foundations of QGSP.

\section{Organization}

This thesis is organized as follows. Chapters \ref{ch:reviewQuat} and \ref{ch:reviewGSP} bring a diverse set of concepts and results from the literature, providing the reader with a sufficient understanding of the quaternion algebra and quaternion Fourier transform (the former), and signal processing from the perspective of graph signals and fractional-order operators (the latter).
Chapter \ref{ch:FrGSO} presents the new fractional graph shift operator, commenting on its properties and applications.
In Chapter \ref{ch:FrQDFT}, a study on the fractionalization of the QDFT is presented, with a new method being proposed, alongside a multiparametric version of the fractional-order transform.
The foundations of QGSP are finally presented in Chapter \ref{ch:QGSP}, with discussions and a few applications with both toy and real world datasets.
The thesis closes with concluding remarks and future works, in Chapter \ref{ch:conclusion}. The Appendix \ref{ch:AppendixA} leads the reader through the demonstration of a relation between systems of quaternion and complex linear equations, useful for a central theorem presented in Chapter \ref{ch:reviewQuat}. As a final note, the reader may notice that many concepts in the chapters dedicated to basic literature review are presented outside formal theorems and propositions. Far from being a conscious decision from the beginning, it happened as an unconscious stylistic choice, prioritizing a steady flow of writing to the detriment of mathematical formalism.

\section{Contributions}
The contributions of this doctorate research, featured in this thesis, are:

\vspace{-1em}
\begin{itemize}[noitemsep]
    \item The proposition and discussion of the fractional graph shift and its application on improving least-squares approximation of linear and shift-invariant ideal filters.
    \item A theorem which proves that the QDFT and the DFT share symmetric eigenvalues, and which enables a novel approach to fractionalizing the quaternion discrete Fourier transform (QDFT).
    \item The definition of a multiparametric fractional QDFT and its application in a novel image encryption scheme.
    \item The proposition of quaternion graph signal processing (QGSP), a new tool designed for specialized scenarios in which both the signal samples and their quantifiable relationship may be expressed as quaternions (thus having at most four dimensions or information sources).
    \item \texttt{gspx}\footnote{The library is available as an open repository: \url{https://github.com/gboaviagem/gspx}.}, an open-source Python library with implementation of the core concepts from QGSP.
\end{itemize}

\section{Publications}

This work has so far published two papers in international journals, one in a global conference, and a book. The paper \textit{``Eigenstructure and fractionalization of the quaternion discrete Fourier transform''}, published in Optik, contains the results of Chapter \ref{ch:FrQDFT}. The 10th volume of IEEE Access received the work \textit{``On the Fractionalization of the Shift Operator on Graphs''}, containing the results of Chapter \ref{ch:FrGSO}. As the understanding of GSP matured during the doctorate research, the author contributed with some chapters to the book ``Processamento de Sinais sobre Grafos: Fundamentos e Aplica{\c c}{\~o}es'', published in the \textit{Notas em Matem\'atica Aplicada} series from the Brazilian society \textit{Sociedade Brasileira de Matemática Aplicada e Computacional (SBMAC)}.

The paper \emph{``The Cosine Number Transform: A Graph Signal Processing Approach''}, presented in the 2019 edition of GlobalSIP, was created during short attempts to further generalize GSP to other algebraic structures, in particular to graphs with edge weights in finite fields. However, it has not evolved into more substantial findings and thus, although it stems from this work, it is not featured in this thesis. The full references of the mentioned publications are:

\begin{itemize}[noitemsep]
    % Citing in MLA Format, extracted from Google Scholar.
    \item Ribeiro, Guilherme and Lima, Juliano. ``The Cosine Number Transform: A Graph Signal Processing Approach''. \textit{2019 IEEE Global Conference on Signal and Information Processing (GlobalSIP)}. IEEE, 2019. DOI: \href{https://doi.org/10.1109/GlobalSIP45357.2019.8969165}{10.1109/GlobalSIP45357.2019.8969165}

    \item Ribeiro, Guilherme B., and Lima, Juliano B. ``Eigenstructure and fractionalization of the quaternion discrete Fourier transform''. \textit{Optik} 208 (2020): 163957. DOI: \href{https://doi.org/10.1016/j.ijleo.2019.163957}{10.1016/j.ijleo.2019.163957}.

    \item Ribeiro, Guilherme B., José R. De Oliveira Neto, and Lima, Juliano B. ``On the Fractionalization of the Shift Operator on Graphs''. \textit{IEEE Access} 10 (2022): 16468-16478. DOI: \href{https://doi.org/10.1109/ACCESS.2022.3149755}{10.1109/ACCESS.2022.3149755}.

    \item Lima, Juliano, et al. \textit{Processamento de Sinais sobre Grafos: Fundamentos e Aplica{\c c}{\~o}es}. Notas em Matem\'atica Aplicada, v. 92. Sociedade Brasileira de Matemática Aplicada e Computacional (SBMAC), 2022.
\end{itemize}

\section{Notation}
The use of symbols is a great way to communicate ideas concisely. However, instead of improving communication, they may make it cumbersome or even not communicate at all if the reader is not adequately familiar with them. Let us go through the main notation in this thesis, to avoid confusion instead of clarity.

Simple numbers, either real-, complex- or quaternion-valued, are represented in italics, as in $q$. Special treatment is given to unit pure quaternions, represented in bold italics, e.g. $\qi$, $\qnu$. Numerical sets (and the related fields or skew-fields) are denoted using blackboard bold face (such as $\mathbb{R}$). Vectors and matrices are written with bold face: the former with lower case (e.g. $\mathbf{v} \in \mathbb{R}^{n}$), and the latter with upper case (e.g. $\mathbf{V} \in \mathbb{R}^{n \times n}$).

The function $\mathrm{diag}(\cdot)$ has context-dependent meaning: when the argument is a matrix, it returns the main diagonal as a column vector; whereas when it is a vector, it produces a diagonal matrix with this vector in its main diagonal. The symbol $\overset{\Delta}{=}$ represents \textit{equality by definition}.

The \textit{symplectic decomposition} of a quaternion number $q$ (or, by extension, of a quaternion-valued vector or matrix) will be \textit{usually} represented as $q = q_1 + q_2 \qj$, with subscripts $1$ and $2$ indicating the simplex and perplex components. When there is risk of mistaking the subscripts for indices (e.g. in summation), the alternative (more bulky) notation will employ superscripts $^{(s)}$ and $^{(p)}$, as in $q = q^{(s)} + q^{(p)} \qj$.
