\pdfbookmark[0]{List of symbols}{symbols}
\chapter*{List of symbols}
\label{sec:symbols}
%\vspace*{-10mm}


\begin{description}[leftmargin=8em,style=nextline]
    \item[$ \mathbf{A} $] Graph weighted adjacency matrix.
    \item[$ \mathbf{L} $] Graph Laplacian matrix.
    \item[$ \mathbf{\Lambda}, \mathbf{J} $] Respectively the eigenvalue matrix (if it exists) and the Jordan matrix of the graph adjacency matrix.
    \item[$ \mathbf{V} $] The (possibly generalized) eigenvector matrix of the graph adjacency matrix.
    \item[$\mathcal{R}e \{ x \}$] Real part of the complex number $x$.
    \item[$\mathcal{I}m \{ x \}$] Imaginary part of the complex number $x$.
    \item[$ \overline{x} $] Conjugate of the complex (or quaternion) $x$. If vectors or matrices are used instead of $x$, the conjugation is realized in each of its entries.
    \item[$\mathbf{M}^T$] Transpose of the matrix $\mathbf{M}$.
    \item[$\mathbf{M}^H$] Conjugate transpose of the matrix $\mathbf{M}$.
    \item[$\mathbb{R}$, $\mathbb{C}$ \textnormal{and} $\mathbb{H}$] Respectively the set of real, complex and (Hamiltonian) quaternion numbers. By association, we may also refer to the respective (skew-)field.
    \item[$\mathbb{F}^n$] Set of the vectors of length $n$ with entries in the (possibly skew) field $\mathbb{F}$.
    \item[$ | \cdot | $] Modulus of a real, complex or quaternion number.
    \item[$ \rchi_A $] Complex adjoint of the quaternion matrix $ \mathbf{A} $.
    \item[$ S(q) $] Scalar part of the quaternion $q$.
    \item[$ \qV(q) $] Vector part of the quaternion $q$.
    \item[$ \qV(\mathbb{H}) $] Set of all pure quaternions (i.e. with null scalar part).
    \item[$ \qi $, $ \qj $ \textnormal{and} $ \qk $] Unit pure quaternions which constitute the canonical basis for the set of pure quaternions.
    \item[$ \mathbb{C}_{\qmu}$] Set of numbers $\{ a + b\qmu \ | \ a, b \in \mathbb{R} \}$ with $\qmu \in \qV(\mathbb{H})$ and $|\qmu| = 1$, isomorphic to the complex numbers.
    \item[$ \Vert \mathbf{x}\Vert_p $] $ \ell_p $-norm of the vector $\mathbf{x} \in \mathbb{C}^n$, defined as $\left(\sum_{k=0}^{n-1} |x_k|^p\right)^{1/p}$.
\end{description}