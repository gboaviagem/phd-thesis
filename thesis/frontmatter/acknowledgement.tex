% !TEX root = ../thesis-example.tex
%
% \pdfbookmark[0]{Agradecimentos}{Agradecimentos}
% \chapter*{Agradecimentos}
% \label{sec:acknowledgement}
%\vspace*{-10mm}

\begin{openingquote}
    Let me stress this point: it is in the simplicity of your \emph{ordinary work}, in the monotonous details of each day, that you have to find the secret, which is hidden from so many, of something great and new: Love. \cite[n. 489]{escriva2016sulco}
\end{openingquote}

At the touch of Love, a divine Person who calls and moves the heart amidst daily activities, the feeling of gratitude is inevitable. Especially here, I thank for the graces granted to carry out this work and bear fruit with the time granted by Him. His kindness knows no end.

I also direct my deep and sincere gratitude to some people who were fundamental in making the development and completion of this work possible.

I thank B\'arbara, my wife, for being the occasion to find our deepest Love every day. In addition to sharing life and the future with me, her dedication in the small things made it possible for me to complete this journey.

I thank my parents and my brother for the education and example with which they nurtured me throughout my life. They gave me the possibility to be who I am, and for that, I will be eternally grateful. Also to my in-laws, sisters-in-law, and close friends, whose strong affection manifests itself in the continuous generosity and support they give to me and my family.

I also thank Prof. Juliano Bandeira, my advisor, who with incredible patience and clarity helped me rediscover the path of research when it seemed so forgotten and challenging. His guidance helped me see the way and gather strength for the last miles of this long-distance race.

I also thank Petrônio Braga and my colleagues at Neurotech for the conversations both inside and outside of work, which in one way or another helped me mature my perspective and not leave behind my academic vein.

Finally, I thank CAPES and the Office of Research and Graduate Studies (in portuguese, Pr\'o-reitoria de Pesquisa e P\'os-gradua\c c\~ao) at UFPE for the financial support during part of this research (under project n. 88882.380374/2019-01), and the Graduate Program in Electrical Engineering at UFPE for the opportunity granted.
