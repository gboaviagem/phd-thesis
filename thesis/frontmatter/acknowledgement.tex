% !TEX root = ../thesis-example.tex
%
% \pdfbookmark[0]{Agradecimentos}{Agradecimentos}
% \chapter*{Agradecimentos}
% \label{sec:acknowledgement}
%\vspace*{-10mm}

% {\normalsize

\begin{quotation}
    \itshape
    % Let me stress this point: it is in the simplicity of your ordinary work, in the monotonous details of each day, that you have to find the secret, which is hidden from so many, of something great and new: Love.
    Insisto: na simplicidade do teu \emph{trabalho habitual}, nos detalhes monótonos de cada dia, tens que descobrir o segredo -- para tantos escondido -- da grandeza e da novidade: o Amor.

    % \noindent --- Saint Josemaria Escriv\'a, \textit{Furrow}, n. 489.
    \noindent --- S\~ao Josemaria Escriv\'a, \emph{Sulco}, n. 489.
\end{quotation}

\vspace{3em}

Ao toque do Amor, uma Pessoa divina que lhe chama e move o cora\c c\~ao em meio \`as atividades corriqueiras, \'e inevit\'avel o sentimento de gratid\~ao. Especialmente, aqui, agrade\c co pelas gra\c cas concedidas para realizar este trabalho e frutificar parte do tempo que Ele me concedeu. Sua benignidade n\~ao tem fim.

Fujo da l\'ingua inglesa, por um momento, para dirigir tamb\'em meus profundos e sinceros agradecimentos a algumas pessoas em nossa l\'ingua materna; amigos e familiares que foram fundamentais para tornar poss\'ivel o desenvolvimento e conclus\~ao deste trabalho.

% Antes deles, no entanto, meu cora\c c\~ao \'e imensamente grato ao bom Deus pelas gra\c cas concedidas para realizar este trabalho e frutificar parte do tempo que Ele me concedeu. Sua benignidade n\~ao tem fim.

Agrade\c co a B\'arbara, minha esposa, por ser a cada dia ocasi\~ao de encontrar o nosso Amor mais profundo. Al\'em de dividir comigo a vida e o futuro, sua dedica\c c\~ao nas pequenas coisas tornaram poss\'ivel que eu conclu\'isse esta jornada.

Agrade\c co a meus pais e meu irm\~ao, pela educa\c c\~ao e exemplo com que me nutriram ao longo da vida. Deram-me a possibilidade de ser quem sou, e por isso serei eternamente grato. Tamb\'em a meus sogros, cunhadas e compadres, cujo carinho forte tal qual la\c cos de sangue se manifesta no suporte t\~ao grande que d\~ao a mim e minha fam\'ilia.

Agrade\c co ao prof. Juliano Bandeira, meu orientador, que com incr\'ivel paci\^encia e lucidez me ajudou a reencontrar o caminho da pesquisa, quando este se mostrava t\~ao esquecido e desafiador. Sua orienta\c c\~ao me ajudou a enxergar o caminho e reunir for\c cas para as \'ultimas milhas dessa corrida de longa dist\^ancia.

Agrade\c co tamb\'em a Petr\^onio Braga e meus colegas da Neurotech, pelas conversas dentro e fora do trabalho, que de uma forma ou de outra me ajudaram a amadurecer meu olhar e n\~ao deixar para tr\'as minha veia acad\^emica.

Agrade\c co, por fim, \`a CAPES e \`a Pr\'o-reitoria de Pesquisa e P\'os-gradua\c c\~ao pelo aux\'ilio financeiro durante parte dessa pesquisa (sob o projeto n. 88882.380374/2019-01), e ao Programa de P\'os-gradua\c c\~ao em Engenharia El\'etrica da UFPE pela oportunidade concedida.
% I am also truly trankful for the scholarship granted by CAPES under the project n. 88882.380374/ 2019-01, which provided an important financial support throughout most of my research.
