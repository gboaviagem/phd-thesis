
% !TEX root = ../thesis-example.tex
%
% \pdfbookmark[0]{Abstract}{Abstract}
% \chapter*{Abstract}
% \label{sec:abstract}
%\vspace*{-10mm}

Signal processing, at its core, is concerned with exploring how different representations of a signal may provide useful ways of manipulating it. These representations may arise from a change in the subjacent algebra over which the signal samples are defined, for example when embedding three- and four-dimensional signals into the quaternion space; or maybe from a different model of the signal domain, as it happened with the development of graph signal processing to deal with network-like data; or even by exploring new linear transforms that map the signal onto a domain in which tasks such as compression, filtering or feature extraction are easier or more efficient.
This thesis traverses exactly through these paths, aiming at answering the question of how to extend graph signal processing to the case in which signals and edge weights are quaternion-valued. It proposes a new set of tools which are a basis for what may be called Quaternion Graph Signal Processing (QGSP) and, as byproducts of the research journey, it contributes to the field of fractional transforms in two fronts: by proposing a new approach to the fractionalization of the quaternion discrete Fourier transform (QDFT), alongside the proposition of its multiparametric version, and by proposing a new fractional graph shift operator (GSO).
Among the main results, we can mention: (1) the polynomial representation of the fractional GSO for arbitrary graphs was obtained, and it was shown that its use in filter design of finite impulse response and linear and shift-invariant (FIR LSI) filters improve the overall filter quality for a given filter length; (2) the new multiparametric fractional QDFT was used to create a holistic encryption scheme for color images with opacity layer, which was shown to provide satisfactorily large key space and key sensitivity; and (3) the main aspects of spectral analysis, filtering and compression in the context of QGSP were formulated, along with extensive practical examples on real-world data computed through a custom-made and open-source Python package.
\vspace{1em}

\noindent
\textbf{Keywords}: quaternions, fractional quaternion discrete Fourier transform, fractional graph shift operator, quaternion graph signal processing.
